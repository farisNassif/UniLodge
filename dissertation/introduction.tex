\chapter*{About this Project}
\paragraph{Abstract}
Housing, and the lack of affordable accommodation has become a hot topic in recent times, especially in relation to students having to endure undesirable living conditions for even more so undesirable rates. Daily, students are commuting great distances to avoid having to endure the financial burden of living at a local level. Students being unable to bear this burden leads to lower admission and attendance rates, an undesirable outcome for both the educational institutions and those looking to attend.

\paragraph{}
The proposed solution to help bridge this issue will be a web application, providing an easily accessible platform for students, and for those living locally or living at a reasonable distance, who may not have the outlet to advertise their spare room or inhabited apartment. 

\paragraph{Authors}
This was developed as a 15 credit project by Faris Nassif and Aaron Burns, final year students of Galway-Mayo Institute of Technology.

\paragraph{Acknowledgements}
The authors would like to acknowledge the project supervisors Dr. John French and Dr. Martin Kenirons for the time and advice they dispensed during the course of the project.

\chapter{Introduction}
This chapter will serve as a prelude to the project. The context, objectives and metrics for success and failure will be defined, followed by a summary of each chapter of the dissertation and its relevance.

\section{Context}
During the decision making process it was decided that the project must be relevant not only to the team but also to peers. The project must also hone existing skills and allow for the natural development of new techniques and processes while also being worthy in scope. 

\paragraph{}
The team felt the issue of affordable student housing, or the lack thereof to be both a familiar and worthy problem to attempt to address. Being students, the team had access to first-hand continuous feedback and input from students on the subject, allowing a more specialized and niche approach, an advantage others who attempt to approach the topic may be in absence of. Following the establishment of the problem area, the team held internal and supervisory discussions on how best to approach the task.

\paragraph{}
\textbf{UniLodge} was the result of much deliberation. UniLodge would serve students and home owners in Galway, allowing for both a practical and streamlined avenue of accommodation advertisement while existing as a simple to use platform for identifying listings that fit the standards and requirements of the student. 
\section{Objectives of the Project}
As previously mentioned, the end goal of the project is to create an application that would help bridge the gap between tenants and students by providing both parties with a platform that would allow for the organization of accommodative housing services specifically for students in the Galway area. To ensure this end goal was reached, objectives were outlined and followed by the team, they are as follows:

\begin{itemize}
    \item Investigate the field by gathering the opinions and ideas of students in relation to the problem area.
    \item Evaluate and investigate the frameworks and tools available for creating a platform independent application.
    \item Create and develop an application based on student feedback that will allow users to arrange or offer lodging services for students.
    \item Incorporate state of the art technologies, frameworks and tools where applicable to ensure the developed application is both easy and simple to navigate while being of applicable standard.
    \item The application will, at a minimum, allow users to register an account, securely login, post listings and communicate with other users via a commenting system.
\end{itemize}

In regards to specific investigative objectives, a brief survey \cite{SURVEY} was issued to fellow students early in the development cycle to help provide an idea of what the target audience would feel to be important in an application of this nature. The results helped shape the objectives and metrics for success or failure. The exact resulting actions taken in response to the outcome of the survey will be discussed throughout the dissertation.

\section{Metrics for Success and Failure}

\section{Dissertation Summary}
This section will contain a brief overview of each chapter outlined in this dissertation.
\subsection{Methodology}
In this chapter, the processes undertaken during the life cycle of the project in regards to planning and development will be outlined. The decisions, thought processes and influential factors leading up to those processes and design implementations will also be described.
\subsection{Technology Review}
A technological review will attempt to encapsulate the technical aspect of the project. This includes the different technologies incorporated, their implementation, why they were implemented and why they were chosen. The benefits of the chosen implementations will be critically analysed and compared with alternatives.
\subsection{System Design}
A detailed explanation of the overall architecture of the project will be provided. Code-snippets and diagrams will be included to help illustrate the inner workings of the application at a high level. Improvements to the system will be identified and potential competitive alternatives will be discussed.
\subsection{System Evaluation}
An evaluation of the software developed in the project will be carried out with the initial project objectives in mind. The final results of the project will be reviewed, including an analysis of areas for improvement and potential changes applicable to the overall system.
\subsection{Conclusion}
To conclude, a brief review will encapsulate the overall system. Key insights will be identified and reflected upon. A final analysis will describe the overall experience and what was learned from the development life-cycle of the project.
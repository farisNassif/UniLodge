\chapter{System Evaluation}
The underlying goal of this project was to develop an easy to use web-application, providing both students searching for accommodation and those looking to share or advertise their rooms with a platform tailored exclusively for them. More precisely, the project goals could be summarized by the following objectives: 

\begin{enumerate}
  \item Investigate the field by gathering the opinions and ideas of students in relation to the problem area.
  
  \item Evaluate and research potential technological inclusions including frameworks, libraries and tools that may see a place within the development cycle.
    
  \item Based on information gathered and research on various state of the art technologies, Build a full stack system that attempts to address the outlined problem domain while collaborating and communicating as a team throughout development
\end{enumerate}

\section{Evaluation of Objectives}
This brief section will describe how each objective was addressed over the course of the project.

\begin{description}
  \item[$\bullet$] \textit{Investigate the field by gathering the opinions and ideas of students in relation to the problem area}. 
  
  \paragraph{}
  \textbf{B}efore project development began, a survey aimed at students was distributed, allowing the team to gain valuable insight on necessary key features, similar applications and by extension set the foundation for which to commence development. 
  
  \paragraph{}
  During development, especially during the earlier stages, students were heavily involved in steering the course of the project, providing key insights. For example, a feature that was intended to be implemented would allow users to 'follow' other users, following testing by students, it was determined that a commenting on system should be a much higher priority.

  \item[$\bullet$] \textit{Evaluate and research potential technological inclusions including frameworks, libraries and tools that may see a place within the development cycle}.
  
  \paragraph{}
  \textbf{R}esearch on a wide array of technologies was conducted which turned out to be highly beneficial  when it came to determining the best course to take with different components of the system. Research and prototyping helped eliminate frameworks and tools that the team felt had no place in the project or wouldn't be as effective in practice as they would be in theory.
  
  \paragraph{}
  As as example, if the project prototype was never developed and research was never conducted, the team may have ultimately considered VueJS as their frontend framework, initiating the development phase only to realise after time that the chosen technologies aren't as compatible with this specific framework as they would be with another. 
  
  \paragraph{}
  Instead, the team were able to take measures and identify the strengths and weaknesses of different technologies working together in unison and choose a stack that could both enhance the development skill of the team and yield a robust and user friendly application.
  
  \item[$\bullet$] \textit{Based on information gathered and research on various state of the art technologies, Build a full stack system that attempts to address the outlined problem domain while collaborating and communicating as a team throughout development}. 

  \paragraph{}
  \textbf{N}ot only does the developed platform allow users to exchange services and communicate with each other, achieving the end-goal of the project, but additional functionality was implemented based on previously gathered information and researched technologies, including login authentication, password security, user profiles and a gallery for images to name a few additional features.
  
  \paragraph{}
  While the application isn't perfect and has it's flaws, which the team are fully aware of, in terms of accomplishing a previously identified goal the application has absolutely accomplished it's objective. 
  
  \paragraph{}
  Unfortunately collaboration and communication between team members is something that cannot be said to be definitively maintained during development, this will be discussed further in the chapter along with system flaws and improvements.
\end{description}

\section{Limitations and Opportunities for Improvement}
While the proposed objectives for the project were achieved, there are still multiple areas within the developed system that can be vastly improved.

\subsection{Server Proxy}
Using a proxy to redirect API calls to Flask is far from optimal. While the application works as is with Node/Express serving the static Angular files, using a proxy to incorporate the Flask API is still only a quick fix for the system. 

\paragraph{}
System testing earlier in the development life cycle would have identified the issue of Flask not correctly serving Angular build files. Angular build files weren't generated until time came to deploy the application, so Flask only ever served development files. Build files were required for Heroku to be able to host the application, meaning the workaround was necessary, but could have been avoided if system tests were incorporated earlier and builds were generated periodically to test deployment. If this was carried out, the issue with Flask would have been able to be addressed much earlier in the development cycle.

\subsection{Testing}
Although periodic unit testing with Karma and Jasmine was conducted and usability tests were incorporated along with acceptance and verification testing to validate the project objectives, overall testing could have been a lot more prevalent throughout the system.

\paragraph{}
During development with the help of fellow students, user stories were written and used to drive some areas 
of development, this approach wasn't incorporated in all aspects of development, however some areas like commenting or browsing components were heavily designed and implemented this way. \newline

\begin{lstlisting}[caption=User Story used as a base for browse functionality]
As a Student looking for Accommodation on a budget
I want to browse Listings in Salthill between 50 and 80 euro
So that I can contact the Poster and inquire about the Listing
\end{lstlisting}

\paragraph{}
Unit Testing wasn't as abundant as it should have been, however during the early stages of development Karma and Jasmine were utilized at an acceptable level to ensure functionality like navigation and intractable components worked as intended. With that said, System Testing was an area of testing that really fell short during the project life-cycle. As mentioned previously, System Testing is something that should have been practiced and seen as a larger priority than it was considered. The biggest issue faced throughout the project was undoubtedly the problem with Flask and Angular build files, should System Testing have been practiced, this could have been identified and addressed at a much earlier stage.

\subsection{Device Portability}
Following deployment, the application was tested on mobile, and while it works, the overall display of pages and components isn't ideal and can be improved upon. This can be solved in future by including some media and device responsive CSS on the most problematic pages.

\paragraph{}
This could have been addressed during development by consistently testing the application on alternative devices, however this was never a priority or an objective. In the future it would absolutely correct the cross-device aesthetic of the application.

\subsection{API Request Speed}
Something that the team had attempted to address at multiple points during the development cycle is the stuttering of some API interactions. This delay would result in components loading faster than the API data could be retrieved, an undesirable feature that isn't pleasant to see in an application. 

\paragraph{}
Different methods like lazy loading and progress, or loading bars were trialed and implemented to attempt to address the inconvenience, however it does still remain a problem with the application with some components more than others.

\section{Overall Evaluation}

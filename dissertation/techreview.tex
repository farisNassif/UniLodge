\chapter{Technology Review}
Over the course of the project life cycle a plethora of frameworks, tools and development applications were available for integration or use with our application. This section aims to discuss the tools and technologies that were ultimately used, why they were chosen and what alternatives were available. 

\section{Initial Considerations}
During the discussion and planning phase goals were outlined and proposed, however how to reach the end point was still very ambiguous. For this reason a lot of time was spent considering different approaches and uncovering the benefits and drawbacks of venturing down a chosen route. This brief section will outline those initially considered approaches.
\subsection{MEAN Stack}
The MEAN Stack combines the best of Javascript based technologies. The Stack is essentially a collection of open source components that provide a streamlined environment for building dynamic web applications. 

\paragraph{}
The MEAN Stack consists of:

\begin{description}
  \item[$\bullet$] MongoDB
  \item[$\bullet$] ExpressJS
  \item[$\bullet$] Angular
  \item[$\bullet$] Node.js
\end{description}

\paragraph{}
Perhaps the greatest attribute of the MEAN Stack for developers is that it's a single language development stack, which can also be one of it's most undesirable attributes depending on the developers Javascript competency \bibitem{mea}. Other 
\paragraph{}


\paragraph{}
For the development team one of the biggest appeals of this stack was 

\section{MEAN Stack}
\subsection{Angular}
\subsection{The Flask Micro-framework}
\subsection{MongoDB}
\subsection{ExpressJS / Node}

\section{Deployment}
\subsection{Heroku}
\subsection{Honcho}
\subsection{Ngrok}

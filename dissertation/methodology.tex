\chapter{Methodology}

\section{Preliminary Research and Project Onset}
The first project meeting began in the final week of September, briefly after the project requirements had been assigned and outlined. An early start was agreed to be something that would greatly benefit the overall development of the project and it was concluded that an idea should be finalized as soon as possible to allow for necessary pre-development research.

\paragraph{}
This section will further explore the aforementioned pre-development process, the influence of the supervisory meetings on this stage, the overall methodical conclusion and it's influence on the initial project architecture.

\subsection{Initial Meeting and Brainstorming}
In the weeks before development began, after the project idea had been finalized, technologies, concepts and potential inclusions were explored and discussed between the team members. A brainstorming phase was conducted on what to incorporate into the project.

\subsubsection{Brainstorming}
The team members met prior to the initial supervisory meeting and discuss potential avenues of exploration during the development phase. To produce effective ideas, questions had to be asked relating to the ultimate goals and objectives of the project, these included:

\begin{itemize}
    \item\textit{\textbf{What research areas should be prioritized before the development phase is initialized?}}
    \item\textit{\textbf{What type of Methodology would best fit our approach?}}
    \item\textit{\textbf{What benefits would different methodologies have when compared to others?}}
\end{itemize}

\subsection{Methodology Consideration}
There were numerous possible methodologies to consider, namely Waterfall, Rapid Application Development and Agile to name a few. Following discussions internally between the team members and talks with supervisors, the list of potential methodologies were shortlisted to both \textbf{Waterfall} and \textbf{Agile}.

\section{Determining the Methodology}
Following the decision to shortlist both Waterfall and Agile, research began on which route would be best to take when considering the scope and overall goals of the project and comparisons between the two were drawn. A brief high-level overview of both methodologies will be included followed by comparisons and a practical analysis of each methodology in relation to the project.

\subsection{Waterfall}
Like most traditional software development models and methodologies, the Waterfall model is based on a series of phases or steps, illustrated in \textit{Figure 2.1}. Waterfall allows progress to be easily measured, the complete scope of the project is known in advance which can be preferred based on the project being undertaken. Since the overall design is finished early in the development cycle, the Waterfall approach is especially effective in projects where various software components need to be designed in parallel \cite{WATERFALL_SURVEY} 

\begin{figure}[H]
	\caption{Waterfall Model Cycle}
	\label{image:myImageName}
	\centering
	\includegraphics[width=0.9\textwidth]{images/waterfall.png}
\end{figure}	

The Waterfall model is arguably the most well known development model, which is likely attributed to how long the model has been around, not to mention it's overall simplicity. Waterfall being an easily understood model naturally means it isn't difficult to manage which is mainly attributed to it's strict requirement definitions that ensures requirements are clearly defined, well received and understood \cite{WATERFALL}. Each phase is processed and undertaken uniformly, meaning phases are completed sequentially and don't overlap \cite{WATERFALL_REVIEW}.

\subsection{Agile}
Initially described in the Manifesto for Agile Software Development, Agile is a an array of principles and methods for project management \cite{AGILE_MANIFESTO}. It is characterized by an iterate approach that allows software to be delivered to the end-user in periodical releases while also employing flexibility, allowing previously defined requirements to shift and project scope to change without significant damage or obstruction on the task currently being undertaken \cite{AGILE}.

\begin{figure}[H]
	\caption{Agile Cycle}
	\label{image:myImageName}
	\centering
	\includegraphics[width=0.9\textwidth]{images/agile.png}
\end{figure}	

The popularity of Agile development methodologies among the software development industry has increased since their introduction in the mid-nineties \cite{AGILE_SURVEY}, with almost 86\% of surveyed international software developers using agile methodologies in their work \cite{AGILE_SURVEY_TWO}.

\subsection{Comparing Agile and Waterfall}
Following supervisory and team meetings, research and analysis, both Agile and Waterfall were compared under the following headings:

\begin{description}
  \item[$\bullet$] Applicability and Compatibility with the Project
  \item[$\bullet$] Requirement Delivery
  \item[$\bullet$] Flexibility
\end{description}

With the establishment of the aforementioned headings, comparisons were drawn between both approaches.

\begin{table}[H]
\begin{tabularx}{\linewidth}{>{\parskip1ex}X@{\kern4\tabcolsep}>{\parskip1ex}X}
\toprule
\hfil\bfseries Agile
&
\hfil\bfseries Waterfall
\\\cmidrule(r{3\tabcolsep}){1-1}\cmidrule(l{-\tabcolsep}){2-2}

%% PROS, seperated by empty line or \par
\begin{compactitem}[-]
\item[+] Strong ability to respond to the changing project requirements.
\item[+] Issues and roadblocks can be detected and addressed rapidly. 
\item[+] Feedback is immediate and helps drive development.
\item[+] Attractive methodologies that would fit the nature and goal of the project, namely \textbf{Kanban} and  \textbf{Scrum}.
\item[+] Encourages frequent end-user involvement, considering college students would be the target audience for the application this Agile benefit is especially attractive. 
\item[$-$] Lack of emphasis on necessary designing and documentation.
\item[$-$] Project may grow ever-larger since there isn't a clearly defined end point.

\end{compactitem}
\par

&

%% CONS, seperated by empty line or \par
\begin{compactitem}[-]
\item[+] Clearly defined and formalized requirements.
\item[+] The software development structure is carefully planned and detailed, minimizing the number of potential issues and roadblocks.
\item[+] Progress is easily measured, the beginning and end points for each phase are fixed allowing easier progress tracking.
\item[$-$] Analysis, planning and requirements definition phase can end up taking time out of actually developing the project.
\item[$-$] Low flexibility level meaning it may be difficult if not impossible to make major changes to the project while in the middle of development, considering the ever changing nature of the defined application this would be troublesome.
\end{compactitem}

\\\bottomrule
\end{tabularx}
\caption{Comparisons drawn between Agile \& Waterfall}
\end{table}

Following comparisons being drawn it became clear to the team an \textbf{Agile} approach would best suit the outlined project. This approach would enable many benefits but those that were the most attractive to bring to the project included:

\begin{enumerate}
  \item \textbf{Interactive user involvement and feedback} - Having the target audience of the application be students while still being in college surrounded by potential users meant this was hugely beneficial.
  \item \textbf{The ability to develop via small periodic releases} - Based on feedback from supervisors and other students the priorities may shift and sway, small incremental changes mean there would be increased flexibility and versatility when it came to adapting to change.
  \item \textbf{Kanban and Scrum Methodologies} - The idea of incremental releases combined with periodic sprints based on workflow defined via the Kanban method was something that the team agreed would improve the overall flow of development following supervisory discussions.
\end{enumerate}

\section{Agile Approach}
Given the ever changing requirements and nature of the project and for additional reasons outlined in the analysis section, the integration of Agile methodologies would be crucial to ensure an effective development cycle.

\paragraph{}
Deciding what Agile methodologies to incorporate into the project was the next step, Kanban and Scrum were two that had already been identified and deemed highly beneficial. Following more research into methodologies including Extreme Programming (XP) and Feature Driven Development (FDD) TODO THE REST OF THIS AT A LATER DATE**

\subsection{Kanban}
Since development began the Kanban methodology was adhered to. The Kanban method is essentially a lean method to manage and improve flow systems. Like Scrum, Kanban is a process intended to help teams work together more effectively and efficiently \cite{KANBAN_SCRUM}. 

\paragraph{}
Kanban allows a team to easily prioritize and visualize the main elements of the project in-progress and also easily delegate tasks based on what work needs to be started, what work is in progress and what has been completed.

\begin{figure}[H]
	\caption{Github Kanban (Scaling needs fixing)}
	\label{image:kanban}
	\centering
	\includegraphics[width=1\textwidth]{images/kanban.png}
\end{figure}

Figure \ref{image:kanban} above consists of the Kanban board used during the development of the project. Initially the team had planned to use Trello \cite{TRELLO} for project tracking, however following discussions with other students and finally supervisory discussions it was found that Github had a built in Kanban function. 

\paragraph{}
Having the ability to track the progress of the project on the same platform as where the project is actually being collaborated on would be a huge benefit, it meant issues and commits could be assigned to  tasks listed on the Kanban board meaning team members could work with more efficiency than if an external tracker like Trello was used.

\subsection{SCRUM}

\section{Version Control}
Initially Github \cite{GITHUB} was the only platform considered for version control coverage, however following discussions with other students, Gitlab \cite{GITLAB} was spoken highly of, so this was another potential consideration. 

\paragraph{}
Github had the home advantage of being a platform that had been used extensively by the team in previous projects, however Gitlab isn't too different and is likely only less popular since it's a newer platform. Both platforms are git based software-development platforms that provide access control and several collaboration features for developers. 

\paragraph{}
The main advantage of GitLab for the team was its open source nature, which allows the hosting of your repositories in your own server free of charge. Gitlab's continuous integration pipelines were also very attractive, Github requires the employment of third-party tools tools like Travis-CI to avail of features like this.

\paragraph{}
The advantages of Gitlab were nice, however following internal discussions it was decided Github would remain the main source of Version Control for the project. The main reason for this being the advantages of Gitlab wouldn't be able to be as utilized as the team would like. Github being familiar and not too different from Gitlab meant there wasn't any deal-breaker that would merit the switch.

\subsection{Github}
A Github repository was setup remotely and used during development to allow for collaboration, code security and to track the progress of the project as well as providing the functionality of managing dependencies and providing alerts if new versions should arise. 

\paragraph{}
The team took advantage of many of the additional features Github has to offer including it's Kanban board, the ability to branch and merge was utilized but could have been used more and the option to open and assign issues to name a few.

\section{Testing}
\subsection{Types of Testing we'll do}

\section{System Architecture}
\subsection{Components of Project}
\subsection{Brief discussion about moving pieces}



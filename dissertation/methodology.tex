\chapter{Methodology}

\section{Preliminary Research and Project Onset}
The first project meeting began in the final week of September, briefly after the project requirements had been assigned and outlined. An early start was agreed to be something that would greatly benefit the overall development of the project and it was concluded that an idea should be finalized as soon as possible to allow for necessary pre-development research.

\paragraph{}
This section will further explore the aforementioned pre-development process, the influence of the supervisory meetings on this stage, the overall methodical conclusion and it's influence on the initial project architecture.

\subsection{Initial Meeting and Brainstorming}
In the weeks before development began, after the project idea had been finalized, technologies, concepts and potential inclusions were explored and discussed between the team members. A brainstorming phase was conducted on what to incorporate into the project.

\subsubsection{Brainstorming}
The team members met prior to the initial supervisory meeting and discuss potential avenues of exploration during the development phase. To produce effective ideas, questions had to be asked relating to the ultimate goals and objectives of the project, these included:

\begin{itemize}
    \item\textit{\textbf{What research areas should be prioritized before the development phase is initialized?}}
    \item\textit{\textbf{What type of Methodology would best fit our approach?}}
    \item\textit{\textbf{What benefits would different methodologies have when compared to others?}}
\end{itemize}

\subsection{Methodology Consideration}
There were numerous possible methodologies to consider, namely Waterfall, Rapid Application Development and Agile to name a few. Following discussions internally between the team members and talks with supervisors, the list of potential methodologies were shortlisted to both \textbf{Waterfall} and \textbf{Agile}.

\subsection{Determining the Methodology}
Following the decision to shortlist both Waterfall and Agile, research began on which route would be best to take when considering the scope and overall goals of the project and comparisons between the two were drawn. A brief high-level overview of both methodologies will be included followed by comparisons and a practical analysis of each methodology in relation to the project.

\subsubsection{Waterfall}
Like most traditional software development models and methodologies, the Waterfall model is based on a series of phases or steps, illustrated in \textit{Figure 2.1}. Waterfall allows progress to be easily measured, the complete scope of the project is known in advance which can be preferred based on the project being undertaken. Since the overall design is finished early in the development cycle, the Waterfall approach is especially effective in projects where various software components need to be designed in parallel \cite{WATERFALL_SURVEY} 

\begin{figure}[h!]
	\caption{Waterfall Model Cycle}
	\label{image:myImageName}
	\centering
	\includegraphics[width=0.9\textwidth]{images/waterfall.png}
\end{figure}	

The Waterfall model is arguably the most well known development model, which is likely attributed to how long the model has been around, not to mention it's overall simplicity. Waterfall being an easily understood model naturally means it isn't difficult to manage which is mainly attributed to it's strict requirement definitions that ensures requirements are clearly defined, well received and understood \cite{WATERFALL}. Each phase is processed and undertaken uniformly, meaning phases are completed sequentially and don't overlap \cite{WATERFALL_REVIEW}.

\subsubsection{Agile}
Initially described in the Manifesto for Agile Software Development, Agile is a an array of principles and methods for project management \cite{AGILE_MANIFESTO}. It is characterized by an iterate approach that allows software to be delivered to the end-user in periodical releases while also employing flexibility, allowing previously defined requirements to shift and project scope to change without significant damage or obstruction on the task currently being undertaken \cite{AGILE}.

\begin{figure}[h!]
	\caption{Agile Cycle}
	\label{image:myImageName}
	\centering
	\includegraphics[width=0.9\textwidth]{images/agile.png}
\end{figure}	

The popularity of Agile development methodologies among the software development industry has increased since their introduction in the mid-nineties \cite{AGILE_SURVEY}, with almost 86\% of surveyed international software developers using agile methodologies in their work \cite{AGILE_SURVEY_TWO}.


\section{Version Control}
Blah...... was cited by \cite{MEAN_STACK} in ... You should refer to images and tables by their label and let latex figure out the numbering for you. E.g. we can refer to the figure on this page as Fig.\ref{image:myImageName} instead of writing "Fig.1"...Blah...... was cited by \cite{MEAN_STACK} in ... You should refer to images and tables by their label and let latex figure out the numbering for you. E.g. we can refer to the figure on this page as Fig.\ref{image:myImageName} instead of writing "Fig.1"...
\subsection{Considerations}
Blah...... was cited by \cite{MEAN_STACK} in ... You should refer to images and tables by their label and let latex figure out the numbering for you. E.g. we can refer to the figure on this page as Fig.\ref{image:myImageName} instead of writing "Fig.1"...

\subsection{Github}

Github was used as the chosen method of version control for the project. A Git repository was setup remotely and used during development to allow for collaboration, code security and to track the progress of the project as well as providing the functionality of managing dependencies and providing alerts if new versions should arise. 




\section{Testing}
\subsection{Types of Testing we'll do}

\section{System Architecture}
\subsection{Components of Project}
\subsection{Brief discussion about moving pieces}


